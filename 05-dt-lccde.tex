\chapter{DT systems as linear constant coefficient difference equations}

A \emph{difference equation} is a relation among combinations of two DT functions and shifted versions of them. Similar to differential equations where the solution is a CT function, the solution to a difference equation is a DT function. For example:
\[                         
y[n+1] + \frac{1}{2}y[n] = x[n] 
\]
is a first order, linear, constant-coefficient difference equation. Given $x[n]$ the solution is a function $y[n]$. We can view this as a representation of a DT system, where $x[n]$ is the input signal and $y[n]$ is the output.

There is a parallel theory to differential equations for solving difference equations. However in this lecture we will focus specifically on the iterative solution of linear, constant-coefficient difference equations and the case when the input is a delta function, as this is all we need for this course.

\section{Definition of linear constant coefficient difference equation}

A \emph{linear}, \emph{constant-coefficient}, difference equation (LCCDE) comes in one of two forms.

\begin{itemize}
  \item Delay form. 
  \[    
  \sum\limits_{k = 0}^N a_k y[n-k] = \sum\limits_{k = 0}^M b_k x[n-k]
  \]
  or
  \[
  a_0y[n] + a_1y[n-1] + \cdots a_N y[n-N] = b_0 x[n] + \cdots b_Mx[n-M]
  \]
  
\item Advance form. Let $n\rightarrow n+N$, then the delay form becomes
  \[    
  \sum\limits_{k = 0}^N a_k y[n+N-k] = \sum\limits_{k = 0}^M b_k x[n+N-k]
  \]
  or 
  \[
  a_0y[n+N] + a_1y[n+N-1] + \cdots a_N y[n] = b_0 x[n+N] + \cdots b_Mx[n+N-M]
  \]
\end{itemize}

The {\it order} of the system is given by $N$. The delay and advance forms are equivalent because the equation holds for any $n$, and we can move back and forth between them as needed by a constant index-shift.

\begin{example}[$N=2$, $M=1$]
  The delay form is
  \[
  a_0y[n] + a_1 y[n-1] + a_2 y[n-2] = b_0 x[n] + b_1 x[n-1]
  \]
  Replacing $n \rightarrow n+2$, the advance form is
  \[
  a_0 y[n+2] + a_1 y[n+1] + a_2 y[n] = b+0 x[n+2] + b_1 x[n+1]
  \]
  $\blacksquare$
\end{example}

It will be convenient to define the operator $E^m$ as shifting a DT function by positive $m$, i.e. $E^m x[n] = x[n+m]$, and the operator $D^m$ as shifting a DT function by negative $m$, i.e. $D^m x[n] = x[n-m]$. These are called the advance and delay operators respectively. Then, the advance form of the difference equation using this operator notation is
\[
a_0y[n+N] + a_1y[n+N-1] + \cdots a_N y[n] = b_0 x[n+N] + \cdots b_Mx[n+N-M]
\]
\[
a_0 E^Ny + a_1E^{N-1}y + \cdots a_N y = b_0 E^{N}x + \cdots b_M E^{N-M}x
\]
Factoring out the advance operators gives
\[
\underbrace{\left(a_0E^N + a_1E^{N-1} + \cdots a_N\right)}_{Q(E)} y = \underbrace{\left(b_M E^{N} + \cdots b_M E^{N-M}\right)}_{P(E)} x
\]
or
\[
Q(E)y[n] = P(E)x[n]
\]

Similarly, the delay form of the difference equation using this operator notation is
\[
a_0y[n] + a_1y[n-1] + \cdots a_N y[n-N] = b_0 x[n] + \cdots b_Mx[n-M]
\]
\[
a_0y[n] + a_1 Dy + \cdots a_N D^N y = b_0 x + \cdots b_MD^M x
\]
Note: The DT delay operator $D$ is similar, but \emph{not} identical to the derivative operator $D$ in CT.

%% To make this example more informative, it may be good to pick DE coefficients not to be symmetric around the middle term. This is because now the two polynomials Q(E) and Q(D) turn out to be the same, whereas that is not the case in general.
%% For example, the DE 
%% y[n]+3y[n-1]+2y[n-2]
%% would have 
%% Q(D)=2D^d+3D+1 and 
%% Q(E)=E^2+3E+2
\begin{example}
  Consider the difference equation
  \[
  3y[n+1] + 4y[n] + 5y[n-1] = 2x[n+1]
  \]
  The advance form would be:
  \[
  3y[n+2] + 4y[n+1] + 5y[n] = 2x[n+2]
  \]
  or using the advance operator
  \[
  \left(3E^2 + 4E + 5\right)y = 2E^2x
  \]
  with $Q(E) = 3E^2 + 4E + 5$ and $P(E) = 2E^2$.\\[1em]
  The delay form would be:
  \[
  3y[n] + 4y[n-1] + 5y[n-2] = 2x[n]
  \]
  or using the delay operator
  \[
  \left(5D^2 + 4D + 3\right)y = 2x
  \]
  with $Q(D) = 5D^2 + 4D + 3$ and $P(D) = 2$.
$\blacksquare$
\end{example}

\section{Iterative solution of LCCDEs}

Difference equations are different (pun!) from differential equations in that they can be solved by manually running the equation forward using previous values of the output and current and previous values of the input, given some initial conditions. This is called an \emph{iterative} solution for this reason.

To perform an iterative solution we need the difference equation in delay form
\[
a_0y[n] + a_1y[n-1] + \cdots a_N y[n-N] = b_0 x[n] + \cdots b_Mx[n-M]
\]
We then solve for the current output $y[n]$
\[
y[n] =  - \left(\frac{a_1}{a_0}y[n-1] + \cdots \frac{a_N}{a_0} y[n-N]\right) + \frac{b_0}{a_0} x[n] + \cdots \frac{b_M}{a_0}x[n-M]
\]

Now lets examine what this expression says in words. To compute the current output $y[n]$ we need the value of the \emph{previous} $N-1$ outputs, the value of the \emph{current} input $x[n]$ and $M-1$ \emph{previous} inputs (and the coefficients). Then we can compute the next output $y[n+1]$ by adding the previous computation result for $y[n]$ to our list of things to remember, and forgetting one previous value of $y$. This can continue as long as we like.

\begin{example}
  Consider the first-order difference equation
  \[
  y[n+1] + y[n] = x[n+1]
  \]
  where $y[-1] = 1$ and $x[n] = u[n]$. We first convert this to delay form
  \[
  y[n] = -y[n-1] + x[n]\; .
  \]
  Then we can compute $y[0]$ as
  \[
  y[0] = -y[-1] + x[0] = -1 + 1 = 0
  \]
  and continuing
  \begin{align*}
  y[1] &= -y[0] + x[1] = 0 + 1 = 1\\
  y[2] &= -y[1] + x[2] = -1 + 1 = 0\\
  y[3] &= -y[2] + x[3] = 0 + 1 = 1\\
  \mbox{etc.}
  \end{align*}
  We can see that this will continue to give the alternating sequence $1,0,1,0,1,\cdots$.
$\blacksquare$
\end{example}

\section{Solution of the homogeneous LCCDE}

Note the iterative solution does not give us (directly) and analytical expression for the output at arbitrary $n$. We have to start at the initial conditions and compute our way up to $n$. We now consider an analytical solution when the input is zero, the solution to the \emph{homogeneous} difference equation
\[
Q(E)\, y = a_0y[n+N] + a_1y[n+N-1] + \cdots a_N y[n] = 0 \; .
\]
given $N$ sequential auxiliary conditions on $y$.

Similar to differential equations, the homogeneous solution depends on the roots of the characteristic equation $Q(E)=0$ whose roots are either real or occur in complex conjugate pairs. Let $\lambda_i$ be the $i$-th root of $Q(E) = 0$, then the solution is of the form
\[
y[n] = \sum\limits_{i=1}^N C_i \lambda_i^{n}
\]
where the parameters $C_i$ are determined from the auxiliary conditions.

For a real system (when the coefficients of the difference equation are real) and when the roots are complex $\lambda_{1,2} = |\lambda|e^{\pm j\beta}$, it is cleaner to assume a form for those terms as
\[
y[n] = C |\lambda|^n\cos(\beta n + \theta)
\]
for constants $C$ and $\theta$.

\begin{example}[First-Order]
  Find the solution to the first-order homogeneous LCCDE
  \[
  y[n+1] + \frac{1}{2}y[n] = 0 \mbox{ with } y[0] = 5 \; .
  \]
  Note $Q(E) = E + \frac{1}{2}$ has a single root $\lambda_1 = -\frac{1}{2}$. Thus the solution is of the form
  \[
  y[n] = C\left( -\frac{1}{2}\right)^n
  \]
  where the parameter $C$ is found using
  \[
  y[0] = C = 5
  \]
  to give the final solution
  \[
  y[n] = 5\left( -\frac{1}{2}\right)^n
  \]
  $\blacksquare$  
\end{example}

\begin{example}[Second-Order, Complex Roots]
  Find the solution to the second-order homogeneous LCCDE
  \[
  y[n+2] + y[n+1] + \frac{1}{2}y[n] = 0 \mbox{ with } y[0] = 1 \mbox{ and } y[1] = 0\; .
  \]
  Note $Q(E) = E^2 + E + \frac{1}{2}$ has a pair of complex roots $\lambda_{1,2} = -\frac{1}{2} \pm j\frac{1}{2}$. Thus the solution is of the form
  \[
  y[n] = C \left|\frac{1}{\sqrt{2}}\right|^n\cos\left(\frac{\pi}{4} n + \theta\right)
  \]
  where the parameters are found using
  \[
  y[0] = C\cos\left(\theta\right) = 1
  \]
  \[
  y[1] = C\frac{1}{\sqrt{2}}\cos\left(\frac{\pi}{4} + \theta\right) = 0
  \]
  This is true when
  \[
  C = \sqrt{2} \mbox{ and } \theta = \frac{\pi}{4} + 2\pi m
  \]
  or
  \[      
  C = -\sqrt{2} \mbox{ and } \theta = -\frac{3\pi}{4} + 2\pi m
  \]
  for any $m\in \mathbb{Z}$ since $\cos$ is periodic in $2\pi$. A final solution is then
  \[
  y[n] = \sqrt{2} \left|\frac{1}{\sqrt{2}}\right|^n\cos\left(\frac{\pi}{4} n + \frac{\pi}{4}\right)
  \]
  $\blacksquare$
\end{example}

See the appendix for a general tehnique to solve for these constants.

\section{Impulse response from LCCDE}

Today our goal is to find the solution to $Q(E)y=P(E)x$ when $x[n] = \delta[n]$ assuming $y[n] = 0$ for $n < 0$, giving the \emph{impulse response} $y[n] = h[n]$. We skip the derivation here and just give a procedure.

\textbf{Step 1:} Let $y_h$ be the homogeneous solution to $Q(E)y_h=0$ for $n > N$.

\textbf{Step 2:} Assume a form for $h[n]$ given by
\[
h[n] = \frac{b_N}{a_N}\delta[n] + y_h[n]u[n]
\]

\textbf{Step 3:} Using the iterative procedure above find the $N$ auxiliary conditions we need by,

\begin{itemize}
\item first, rewrite the equation in delay form and solve for $y[n]$,
\item then let $x[n] = \delta[n]$ and manually compute $h[0]$ assuming $h[n] = 0$ for $n < 0$,
\item repeating the previous step for $h[1]$, continuing up to $h[N-1]$.
\end{itemize}

\textbf{Step 4:} Using the auxillary conditions in step 3, solve for the constants in the solution $h[n]$ from step 2.

\begin{example}

  Find the impulse response of the system given by
  \[
  y[n+2] -\frac{1}{4}y[n+1] -\frac{1}{8}y[n]= 2x[n+1]
  \]

  For step 1 we solve the equation
  \[
  y_h[n+2] -\frac{1}{4}y_h[n+1] -\frac{1}{8}y_h[n] = 0
  \]
  which is of the form
  \[
  y_h[n] = C_1 \left( -\frac{1}{4}\right)^n + C_2 \left( \frac{1}{2}\right)^n
  \]
  since the roots of $Q(E) = E^2 - \frac{1}{4}E - \frac{1}{8}$ are $-\frac{1}{4}$ and $\frac{1}{2}$.

  For step 3, we find the auxiliary conditions needed to find $C_1$ and $C_2$ by rewriting the original equation in delay form and solving for $y[0]$ and $y[1]$ when $x[n] = \delta[n]$.
  \[
  y[n] = \frac{1}{4}y[n-1] + \frac{1}{8}y[n-2] + 2x[n-1]
  \]    
  Let $x[n] = \delta[n]$ and manually compute $y[0]$ assuming $y[n] = 0$ for $n < 0$
  \[
  y[0] = \frac{1}{4}\underbrace{y[0-1]}_{0} + \frac{1}{8}\underbrace{y[0-2]}_{0} + 2\underbrace{\delta[0-1]}_{0} = 0
  \]
  Repeat for $y[1]$
  \[
  y[1] = \frac{1}{4}\underbrace{y[1-1]}_{0} + \frac{1}{8}\underbrace{y[1-2]}_{0} + 2\underbrace{\delta[1-1]}_{1} = 2
  \]  
  Now we find the constants using step 4
  \[
  h[0] = C_1  + C_2  = 0
  \]
  \[
  h[1] = C_1 \left( -\frac{1}{4}\right) + C_2 \left( \frac{1}{2}\right) = 2
  \]
  which gives $C_1 = -\frac{8}{3}$ and $C_2 = \frac{8}{3}$. Thus the final impulse response is
  \[
  h[n] = \frac{b_N}{a_N}\delta[n] + y_h[n]u[n] = -\frac{8}{3}\left( -\frac{1}{4}\right)^nu[n] + \frac{8}{3}\left( \frac{1}{2}\right)^n u[n]
  \]
  since $b_N = 0$.
$\blacksquare$
\end{example}

  Note we can confirm our closed-form result in the previous example, for a few values of $n$, by iteratively solving the difference equation
  \[
    h[0] = \frac{1}{4}\underbrace{h[0-1]}_{0} + \frac{1}{8}\underbrace{h[0-2]}_{0} + 2\underbrace{\delta[0-1]}_{0} = 0
  \]
  \[
      h[1] = \frac{1}{4}\underbrace{h[1-1]}_{0} + \frac{1}{8}\underbrace{h[1-2]}_{0} + 2\underbrace{\delta[1-1]}_{1} = 2
    \]
    \[
      h[2] = \frac{1}{4}\underbrace{h[2-1]}_{2} + \frac{1}{8}\underbrace{h[2-2]}_{0} + 2\underbrace{\delta[2-1]}_{0} = \frac{1}{2}
    \]
    \[
      h[3] = \frac{1}{4}\underbrace{h[3-1]}_{\frac{1}{2}} + \frac{1}{8}\underbrace{h[3-2]}_{2} + 2\underbrace{\delta[2-1]}_{0} = \frac{3}{8}
    \]
    and comparing to our closed-form solution a the same values of $n$
    \[
    h[0] = -\frac{8}{3} + \frac{8}{3} = 0
    \]
    \[
    h[1] = -\frac{8}{3}\left( -\frac{1}{4}\right) + \frac{8}{3}\left( \frac{1}{2}\right) = 2
    \]
    \[
    h[2] = -\frac{8}{3}\left( -\frac{1}{4}\right)^2 + \frac{8}{3}\left( \frac{1}{2}\right)^2 = \frac{1}{2}
    \]
    \[
    h[3] = -\frac{8}{3}\left( -\frac{1}{4}\right)^3 + \frac{8}{3}\left( \frac{1}{2}\right)^3 = \frac{3}{8}
    \]

\begin{example}
  Find the impulse response of the system given by
  \[
  y[n+1] - \frac{1}{2}y[n] = x[n+1] + x[n]
  \]

  In step 1 we note the solution to $Q(E)y[n] = 0$ is of the form
  \[
  y_h[n] = C\left( \frac{1}{2}\right)^n
  \]
  From step 2 we note $b_N = 1$ and $a_N = -\frac{1}{2}$, so that
  \[
  h[n] = -2\delta[n]  +  C\left( \frac{1}{2}\right)^n\, u[n]
  \]
  In step 3 we manually find $h[0]$
  \begin{align*}
    y[n] &= \frac{1}{2}y[n-1] + x[n] + x[n-1]\\
    h[n] &= \frac{1}{2}y[n-1] + \delta[n] + \delta[n-1]\\
    h[0] &= 0 + 1 + 0 = 1
  \end{align*}
  And in step 4 we solve for $C$
  \[
  h[0] = -2  +  C = 1 \mbox{ implies } C = 3
  \]
  to give
  \[
  h[n] = -2\delta[n]  +  3\left( \frac{1}{2}\right)^n\, u[n]
  \]
  $\blacksquare$
\end{example}

