\chapter{Linear time invariant DT systems}

\section{DT system representations}

We can mathematically represent, or model, DT systems multiple ways.

\begin{itemize}
\item purely mathematically - in time domain we will use

  \begin{itemize}
  \item linear, constant coefficient difference equations, e.g.
    \[
    y[n] = a y[n-1] + b y[n-2] + x[n]
    \]
  \item DT impulse response $h[n]$
  \end{itemize}
\item purely mathematically - in frequency domain we will use
  \begin{itemize}
  \item frequency response
  \item transfer function (complex frequency, covered in ECE 3704)
  \end{itemize}
\item graphically, using a mixture of math and block diagrams
\end{itemize}

\section{System properties and classification}

Choosing the right kind of system model is important. Here are some important properties that allow us to broadly classify systems.
\begin{itemize}
\item Memory
\item Invertability
\item Causality
\item Stability
\item Time-invariance
\item Linearity
\end{itemize}

Let's define each it turn.

\subsection{Memory}
The output of a DT system with memory depends on previous or future inputs and is said to be {\it dynamic}. Otherwise the system is memoryless or {\it instantaneous}, and the output $y[n]$ at index $n$ depends only on $x[n]$.
For example:
\[
y[n] = 2x[n]
\]
is a memoryless system, while
\[
y[n+1] + y[n] = x[n]
\]
has memory. To  see this, write the difference equation in recursive form
\[
y[n] = -y[n-1] + x[n-1]
\]
and we see explicitly the current output $y[n]$ depends on past values of output and input.

\subsection{Invertability}

A system is invertible if there exists a system that when placed in series with the original recovers the input.
\[
x[n] \mapsto{T} y[n] \mapsto{T^{-1}} x[n]
\]
where $T^{-1}$ is the inverse system of $T$. For example, consider a system
\[
x[n] \mapsto y[n] = \sum\limits_{m=-\infty}^{n} x[m]
\]
and a system
\[
y[n] \mapsto z[n] = y[n] - y[n-1]
\]
The combination in series $x[n] \mapsto y[n] \mapsto z[n] = x[n]$, since
\[
z[n] = y[n] - y[n-1] = \sum\limits_{m=-\infty}^{n} x[m] - \sum\limits_{m=-\infty}^{n-1} x[m] = x[n]
\]
i.e. the difference undoes the accumulation.

\subsection{Causality}
A DT system is causal if the output at index $n$ depends on the input for index values at or before $n$:
\[
y[n] \;\text{depends on}\; x[m] \;\text{for} \; m \leq n
\]
While all physical CT systems are causal, practical DT systems may not be since we can use memory to "shift time". For CT systems we cannot store the infinite number of values between two time points $t_1$ and $t_2$, but we can store the $n_2-n_1$ values of a DT system between between two indices $n_1$ and $n_2$ (assuming infinite precision).

\begin{example}
Consider a DT system whose difference equation is
\[
y[n] = -x[n-1] + 2x[n] - x[n+1]
\]
We see the current output $y[n]$ depends on a "future" value of the input $x[n+1]$. Thus the system \textbf{is not} causal. In practice we can shift the difference equation to
\[
y[n-1] = -x[n-2] + 2x[n-1] - x[n]
\]
and then delay the output by one sample to get $y[n]$.
\end{example}

\begin{example}
Consider a DT system whose difference equation is
\[
y[n] = -y[n-1] + 2x[n]
\]
We see the current output $y[n]$ depends on a "past" value of the output $y[n-1]$ and the current input $x[n]$. Thus the system \textbf{is} causal. In practice we can immediately compute $y[n]$ with no delay. 
\end{example}

\subsection{Stability}

A DT system is (BIBO) stable if applying a bounded-input (BI)
\[
\left|x[n]\right| < \infty \; \forall \; n
\]
results in a bounded-output (BO) $x[n] \mapsto y[n]$ and 
\[
\left|y[n]\right| < \infty \; \forall \; n
\]
Note, bounded in practice is limited by the physical situation, e.g. the number of bits used to store values.

For example, a DT system described by the LCCDE
\[
y[n+1] - 2 y[n] = x[n+1]
\]
is unstable because the solution $y[n]$ will have one term of the form $\left( 2\right)^n$, for most non-zero inputs $x[n]$ or any non-zero initial condition, that grows unbounded as $n$ increases.

\subsection{Time-invariance}
A DT system is time(index)-invariant if, given
\[
x[n] \mapsto y[n]
\]
then an index-shift of the input leads to the same index-shift in the output
\[
x[n-m] \mapsto y[n-m]
\]

An important example is a DT system described by a LCCDE, e.g.
\[
y[n+1] - \frac{1}{2} y[n] = x[n+1]
\]
or in recursive form
\[
y[n] = \frac{1}{2} y[n-1] + x[n]
\]

If we index shift the input $x[n - m]$ we replace $n$ by $n-m$ and the difference equation becomes
\[
y[n-m+1] - \frac{1}{2} y[n-m] = x[n-m+1]
\]
which has the same solution shifted by $m$
\[
y[n-m] = \frac{1}{2} y[n-m -1] + x[n-m]
\]

If a coefficient depends on $n$ however, e.g
\[
y[n+1] - \frac{n}{2} y[n] = x[n+1]
\]
so that it is no longer LCC then the solution depends on $m$ and the system is no longer time-invariant.

\subsection{Linearity}

A DT system is linear if the output due to a sum of scaled individual inputs is the same as the scaled sum of the individual outputs with respect to those inputs. In other words given
\[
x_1[n] \mapsto y_1[n] \;\text{and}\; x_2[n] \mapsto y_2[n]
\]
then
\[
a x_1[n] + b x_2[n] \mapsto a y_1[n] + b y_2[n]
\]
for constants $a$ and $b$.
Note this property extends to sums of arbitrary signals, e.g. if
\[
x_i[n] \mapsto y_i[n] \; \forall\; i \in [1 \cdots N]
\]
then given $N$ constants $a_i$, if the system is linear
\[
\sum\limits_{i = 1}^N a_i x_i[n] \mapsto \sum\limits_{i = 1}^N a_i y_i[n] 
\]
This is a very important property, called {\it superposition}, and it simplifies the analysis of systems greatly.

An important non-linear system is that is described by a LCCDE with non-zero auxiliary conditions at some $n_0$, $y[n_0] = y_0$. As in CT, such systems will have a term in it's solution that depends on $y_0$. Given two inputs, each individual response will have that term in it, so their sum has double that term. However the response due to the sum of the inputs would again only have one and the sum of the responses would not be the same as the response of the sum. Such a system cannot be linear. Thus the system must be "at rest" before applying the input in order to be a linear system.

\section{Stable LTI Systems}

The remainder of this course is about stable, linear, time-invariant (LTI) systems. As we have seen in DT such systems can be described by a LCCDE with zero auxiliary (initial) conditions (the system is \emph{at rest}). 

We have seen previously how to find the impulse response, $h[n]$, of such systems. We now note some relationships between the impulse response and the system properties described above.

\begin{itemize}
\item If a system is memoryless then $h[n] = C \delta[n]$ for some constant $C$.
\item If a system is causal then  $h[n] = 0$ for $n < 0$.
\item If a system is BIBO stable then
  \[
  \sum\limits_{-\infty}^{\infty} |h[n]| < \infty
  \]
\end{itemize}

