\chapter{CT Systems as Linear Constant Coefficient Differential Equations}

Recall a system is a transformation of signals, turning the input signal into the output signal. While this might seem like a new concept to you, you already know something about them from your differential equations course, i.e. MATH 2214 and your circuits course.

For example, consider the following circuit:
\begin{center}
  \begin{circuitikz}[american voltages,scale=0.8, every node/.style={transform shape}]
    \draw
    (0,2.3) to[battery, l=$1\mbox{ VDC}$] (0,0)
    (3,2) node[spdt,xscale=-1,yscale=-1,anchor=in] (Sw) {}
    (0,2.3) to[short] (Sw.out 2)
    (Sw.out 1) to[short] (1.8,0)
    (0,0) to[short, -o] (8,0)
    (3,2) to[short] (4,2)
    (4,2) to[open, v=$x(t)$] (4,0)
    (4,2) to[R, l=$R$] (6,2)
    (6,2) to[short, -o] (8,2)
    (6,0) to[C, l=$C$] (6,2)
    (8,2) to[open, v=$V_C(t)$] (8,0);
  \end{circuitikz}
\end{center}
where the switch moves position at $t = 0$. The governing equation for the circuit when $t < 0$ is
\[
\frac{dV_c}{dt}(t) + \frac{1}{RC}V_c(t) = 0
\]
a \emph{homogeneous} differential equation of first-order. From a DC analysis, the initial condition on the capacitor voltage is $V_C(0^-) = 0$, so there is no current flowing prior to $t = 0$ and the solution is $V_C(t) = 0$ for $t < 0$.

After the switch is thrown, the governing equation for the circuit when $t \geq 0$ is
\[
\frac{dV_c}{dt}(t) + \frac{1}{RC}V_c(t) = \frac{1}{RC}
\]
Since the voltage across the capacitor cannot change instantaneously $V_C(0^-) = V_C(0^+) = 0$, giving the auxillary condition necessary to solve this equation, which has the form
\[
V_C(t) = A + Be^{-\frac{1}{RC}t}
\]
Using the auxillary condition we find
\[
V_C(0) = A + Be^{-\frac{1}{RC}0} = A + B = 0 \mbox{ which implies } B = -A 
\]
Subsitution back into the differential equation and equating the coefficients gives $A = 1$. Thus the voltage for $t \geq 0$ is
\[
V_C(t) = 1 - e^{-\frac{1}{RC}t}
\]

Suppose we consider the voltage after the switch as the input signal $x(t)$ to the system composed of the series RC. As we have seen previously a mathematical model of the switch is the unit step $x(t) = u(t)$. Suppose we consider the capacitor voltage at the outut of the system, so that $y(t) = V_C(t)$. Then we can consider the system to be represented by the \emph{linear, constant-coefficient differential equation}
\[
\frac{dy}{dt}(t) + \frac{1}{RC}y(t) = \frac{1}{RC}x(t)
\]
where $x(t) = u(t)$ and the solution $y(t)$ is the \emph{step response}
\[
y(t) = \left(1 - e^{-\frac{1}{RC}t}\right)u(t)
\]

As we will see later this representation of systems is central to the course, so we take some time here to review the solution of such equations.
 
\section{Solving Linear, Constant Coefficient Differential Equations}

A linear, constant coefficient (LCC) differential equation is of the form
\[
a_0\, y + a_1\, \frac{dy}{dt} + a_2\, \frac{d^2y}{dt^2} + \cdots + a_N\, \frac{d^Ny}{dt^N}  = b_0\, x + b_1\, \frac{dx}{dt} + bb_2\, \frac{d^2x}{dt^2} + \cdots + b_M\, \frac{d^My}{dt^M}
\]
which can be written compactly as
\[
\sum\limits_{k = 0}^{N} a_k\, \frac{d^ky}{dt^k} = \sum\limits_{k = 0}^{M} b_k\, \frac{d^kx}{dt^k}
\]

It is helpful to clean up this notation using the derivative operator $D^n = \frac{d^n}{dt^n}$. For example
$D^2y = \frac{d^2y}{dt^2}$ and $D^0 y= y$. To give for form as
\[
\sum\limits_{k = 0}^{N} a_k\, D^k y = \sum\limits_{k = 0}^{M} b_k\, D^k x
\]

We can factor out the derivative operators
\[
a_0y + a_1Dy + a_2D^2y + \cdots + a_ND^Ny  = b_0\, x + b_1\, Dx + b_2\, D^2x + \cdots + b_M\, D^M x
\]
\[
\underbrace{\left(a_0 + a_1D + a_2D^2 + \cdots + a_ND^N\right)}_{\text{Polynimial in } D, Q(D)} y = \underbrace{\left(b_0 + b_1 D + b_2 D^2 + \cdots + b_M D^M\right)}_{\text{Polynimial in } D, P(D)} x
\]
to give:
  
\[
Q(D)y = P(D)x
\]
You learned how to solve these in differential equations (Math 2214) as
\[
y(t) = y_\text{h}(t) + y_\text{p}(t)
\]

The term $y_\text{h}(t)$ is the solution of the homogeneous equation
\[
Q(D)y = 0
\]
Given the $N-1$ auxillary conditions $y(t_0) = y_0$, $Dy(t_0) = y_1$, $D^2y(t_0) = y_2$, up to $D^{N-1}y(t_0) = y_{N-1}$.

The term $y_\text{p}(t)$ is the solution of the particular equation
\[
Q(D)y = P(D)x
\]
for a given $x(t)$.

Rather than recapitulate the solution to $y_\text{h}(t)$ and $y_\text{p}(t)$ in the general case we focus on the homogeneous solution $y_\text{h}(t)$ only. The reason is that we will use the homogeneous solution to find the impulse response below and take a different approach to solving the general case for an arbitrary input using the impulse response and convolution (next week).

To solve the homogenous system:

\textbf{Step 1:} Find the \emph{characteristic equation} by replacing the derivative operators by powers of an aribrary complex variable $s$.
\[
Q(D) = a_0 + a_1D + a_2D^2 + \cdots + a_ND^N
\]
becomes
\[
Q(s) = a_0 + a_1s + a_2s^2 + \cdots + a_Ns^N
\]
a polynomial in $s$ with $N$ roots $s_i$ for $i = 1, 2, \cdots, N$ such that
\[
(s - s_1)(s-s_2)\cdots(s-s_N) = 0
\]

\textbf{Step 2:} Select the form of the solution, a sum of terms corresponding to the roots of the characteristic equation.

\begin{itemize}
\item For a real root $s_1\in \mathbb{R}$ the term is of the form
  \[
  C_1 e^{s_1 t}.
  \]
\item For a pair of complex roots (they will always be in pairs) $s_{1,2} = a \pm jb$ the term is of the form
  \[
  C_1 e^{s_1 t} + C_2 e^{s_2 t} = e^{a t}\left(C_3\cos(bt) + C_4\sin(bt)\right) = C_5 e^{a t}\cos(bt + C_6).
  \]
\item For a repeated roots $s_1$, repeated r times, the term is of the form
  \[
  e^{s_1 t} (C_0 + C_1 t + \cdots + C_{r-1} t^{r-1}).\]
\end{itemize}

\textbf{Step 3:} Solve for the unknown constants in the solution using the auxillary conditions. 

We now examine two common special cases, when $N=1$ (first-order) and when $N=2$ (second-order).

\subsection{First-Order Homogeneous LCCDE}

Consider the first order homogeneous differential equation
\[
\frac{dy}{dt}(t) + ay(t) = 0 \mbox{ for } a \in \mathbb{R}
\]
The characteristic equation is given by
\[
s + a = 0
\]
which has a single root $s_1 = -a$. The solution is of the form
\[
y(t) = Ce^{s_1 t} = Ce^{-a t} 
\]
where the constant $C$ is found using the auxillary condition $y(t_0) = y_0$.

\textit{Example}: Consider the homogeneous equation
\[
\frac{dy}{dt}(t) + 3y(t) = 0 \mbox{ where } y(0) = 10
\]
The solution is
\[
y(t) = Ce^{-3 t} 
\]
To find $C$ we use the auxillary condition
\[
y(0) = Ce^{-3 \cdot 0} = C = 10
\]
and the final solution is
\[
y(t) = 10e^{-3 t} 
\]
\subsection{Second-Order Homogeneous LCCDE}

Consider the second-order homogeneous differential equation
\[
\frac{d^2y}{dt^2}(t) + a\frac{dy}{dt}(t) + by(t) = 0 \mbox{ for } a,b \in \mathbb{R}
\]
The characteristic equation is given by
\[
s^2 + as + b = 0
\]

Let's look at several examples to illustrate the functional forms.

Example 1:
\[
\frac{d^2y}{dt^2}(t) + 7\frac{dy}{dt}(t) + 10y(t) = 0 
\]
The characteristic equation is given by
\[
s^2 + 7s + 10 = 0
\]
which has roots $s_1 = -2$ and $s_2 = -5$. Thus the form of the solution is
\[
y(t) = C_1e^{-2t} + C_2e^{-5t}
\]

Example 2:
\[
\frac{d^2y}{dt^2}(t) + 2\frac{dy}{dt}(t) + 5y(t) = 0 
\]
The characteristic equation is given by
\[
s^2 + 2s + 5 = 0
\]
which has complex roots $s_1 = -1+j2$ and $s_1 = -1-j2$. Thus the form of the solution is
\[
y(t) = e^{-t}\left(C_1\cos(2t) + C_2\sin(2t)\right)
\]

Example 3:
\[
\frac{d^2y}{dt^2}(t) + 2\frac{dy}{dt}(t) + y(t) = 0 
\]
The characteristic equation is given by
\[
s^2 + 2s + 1 = 0
\]
which has a root $s_1 = -1$ repeated $r=2$ times. Thus the form of the solution is
\[
y(t) = e^{-t}\left(C_1 + C_2t\right)
\]

In each of the above cases the constants, $C_1$ and $C_2$, are found using the auxillary conditions $y(t_0)$ and $y\prime(t_0)$.

\section{Finding the impulse response of a system described by a LCCDE}

As we will see next week an important response of a system is the one that corresponds to an impulse input, i.e. the \emph{impulse response} $y(t) = h(t)$ when $x(t) = \delta(t)$. Thus we focus here on a recipe for solving LCCDEs for this special case when $M \leq N$. We will skip the derivation of why this works.

Our goal is to find the solution to $Q(D)y = P(D)x$ when $x(t)=\delta(t)$.

\textbf{Step 1:} Rearrange the LCCDE so that $a_N = 1$, i.e. divide through by $a_N$ to put it into a standard form.\\

\textbf{Step 2:} Let $y_h(t)$ be the homogeneous solution to $Q(D)y_h = 0$ for auxillary conditions
  \[
    D^{N-1}y_h(0^+) = 1 \; , \; D^{N-2}y_h(0^+) = 0 \; , \; \text{etc.} \; y_h(0^+) = 0 
    \]
    
\textbf{Step 3:} Assume a form for $h(t)$ given by:
  \[
  h(t) = \underbrace{b_N\delta(t)}_{=0 \text{ unless } N=M} + \underbrace{\left[ P(D)y_h\right]}_{\text{apply } P(D) \text{ to } y_n(t)}u(t)
  \]

Recall from above the homogeneous solution depends on the roots of the characteristic equation $Q(D) = 0$.

\begin{itemize}
\item roots are either real, or
\item roots occur in complex conjugate pairs, or
\item repeated roots.
\end{itemize}

Example 1: Find the impulse response of the LCCDE
\[
2\frac{dy}{dt}(t) + 2y(t) = 2x(t)
\]
In the standard for the LCCDE is
\[
\frac{dy}{dt}(t) + y(t) = x(t)
\]
The characteristic equation is given by
\[
s + 1 = 0
\]
which has a single root $s_1 = -1$. The solution is of the form
\[
y_h(t) = Ce^{-t} 
\]
with the special auxillary condition $y(0) = 1$, so that
\[
y_h(t) = e^{-t} 
\]
Since $P(D) = 1$ and $N = 1 \neq M = 0$ the impulse response is
\[
h(t) = \underbrace{b_N\delta(t)}_{=0} + \left[ \underbrace{P(D)}_{1}y_h(t)\right]u(t) = e^{-t}u(t)
\]

Example 2: Find the impulse response of the LCCDE
\[
\frac{dy}{dt}(t) + y(t) = \frac{dx}{dt}(t) + x(t)
\]
It is already in the standard form. The homogeneous solution is the same as in Example 1,
\[
y_h(t) = e^{-t} 
\]
however now $M = N = 1$ with $b_1 = 1$ and $P(D) = D+1$. Thus, the impulse response is
\[
h(t) = \underbrace{b_N}_{=1}\delta(t) + \left[ \underbrace{P(D)}_{D+1}y_h(t)\right]u(t) = \delta(t) + \left\{[D+1]e^{-t}\right\}u(t) = \delta(t) + [- e^{-t} + e^{-t}]u(t) = \delta(t) 
\]

Example 3: Find the impulse response of the LCCDE
\[
\frac{d^2y}{dt^2}(t) + 7\frac{dy}{dt}y(t) + 10y(t) = x(t) 
\]
It is already in the standard form. The characteristic equation is given by
\[
s^2 + 7s + 10 = 0
\]
which has roots $s_1 = -2$ and $s_2 = -5$. Thus the form of the solution is
\[
y_h(t) = C_1e^{-2t} + C_2e^{-5t}
\]
The special auxillary conditions are $y_h(0) = 0$ and $y^\prime_h(0) = 1$. Using these conditions
\[
y_h(0) = C_1e^{-2t} + C_2e^{-5t} |_{t = 0} = C_1 + C_2 = 0
\]
\[
y^\prime_h(0) = -2C_1e^{-2t} - 5C_2e^{-5t} |_{t = 0} = -2C_1 -5C_2 = 1
\]
Solving for the constants gives $C_1 = \frac{1}{3}$ and $C_2 = -\frac{1}{3}$. Since $P(D) = 1$ and $N = 2 \neq M = 0$ the impulse response is
\[
h(t) = \underbrace{b_N\delta(t)}_{=0} + \left[ \underbrace{P(D)}_{1}y_h(t)\right]u(t) = \frac{1}{3} e^{-2t}u(t) - \frac{1}{3} e^{-5t}u(t)
\]

